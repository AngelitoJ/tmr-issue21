\documentclass{tmr}

\usepackage{amsmath}
\usepackage{listings}
\usepackage[euler]{textgreek}
\usepackage{braket}

\lstset{language=Haskell,
      basicstyle=\footnotesize,
        keywordstyle=\color{black}\bfseries,                                         
        frame=tb,
        showstringspaces=false,
        breaklines=true,
        morekeywords={|getZipList,ZipList}     
}

\newcommand{\inftyint}{\int_{-\infty}^{+\infty}} 

\title{Haskell ab initio: the Hartree-Fock Method in Haskell}
\author{Felipe Zapata\email{felipe.zapata@edu.uah.es}}
\author{Angel J. Alvarez\email{a.alvarez@uah.es}}


\begin{document}

\begin{introduction}

Scientific computing is a transversal subject where professionals of
many fields join forces to answer questions about the behaviour of
Nature using a variety of models.
%
In this area, Fortran has been king for many years.
%
It is now time to end Fortran's tyrannical reign!
%
It is time to use a language which offers a high level of abstraction; a
language which allows a straightforward translation of equations to code.
%
It is time to use a language which has appropriate tools for parallelism
and concurrency.
%
Haskell is our language of choice: its levels of abstraction lead to a
brief, elegant and efficient code.
%
In this article, we will describe a minimal but complete Haskell
implementation of the Hartree-Fock method, which is widely used in
quantum chemistry and physics for recursively calculating the
eigenvalues of the quantized levels of energy of a molecule and the
eigenvectors of the wave function.
%
Do not be afraid about the formidable name; we will skip most of the
technical details and focus on the Haskell programming.

\end{introduction}


\section{Joining two worlds}

Haskell and its underlying theory have made us ask ourself some
irresistible questions: have those equations written in the piece of
paper the same mathematical meaning of those that we have implemented in
Fortran?
%
If programming is as much mathematical as it is artistic creation, then
why are we still working with such twisted and ugly ideas?
%
You ask the same questions to your workmates and professors, and after
while working locked in your office, you will find out that an angry mob
of Fortran programmers is waiting outside.
%
After all, you dared to say that a pure and lazy functional language is
the future of programming in science!

While waiting for the mob to get into our office, we will describe the
Jacobi algorithm for calculating the eigenvalues and eigenvectors of a
symmetric square matrix using the repa library.
%
Then, equipped with this useful recursive function, we will see some
basic details of the Hartree-Fock methodology and the self-consistent
field (SCF) procedure for iteratively computing the eigenvalues and
eigenvectors of a molecular system.
%
In doing so, we will try to connect the simulation ideas with 
the powerful abstraction system of Haskell.
%
We note that there is an excellent collection of modules written by Jan
Skibinski for quantum mechanics and mathematics, but the approach used
in those modules is different from ours~\cite{Skibinski}.


\section{The Jacobi Algorithm}

The Jacobi Algorithm is a recursive procedure for calculating all of the
eigenvalues and eigenvectors of a symmetric matrix.
%
We briefly describe the eigenvalue problem; for a more detailed
treatment, see~\cite{eigenvalue}.
%
The standard matrix eigenvalue problem seeks to find matrices $x$ and $\lambda$
such that:

\[\mathbf{Ax} = \lambda \mathbf{x}  \]

(The \textlambda\ is a diagonal matrix of the eigenvalues; not a
function abstraction!)
%
The Jacobi algorithm is based on applying a transformation of the form

\[\mathbf{A^*x^*} = \lambda \mathbf{x^*}  \]

where

\[\mathbf{x^*} = \mathbf{Rx} \]
\[\mathbf{A^*} = \mathbf{R^TAR} \]

The transformation is applied to the original problem in such a way that
the new expression obtained has the same eigenvalues and eigenvectors,
but contains a matrix {\textbf A\textsuperscript{*}} which is diagonal.
%
The matrix {\bf R} is called the Jacobi rotation matrix~\cite{Jacobi},
which is an orthogonal matrix ({\textbf R\textsuperscript{-1}}= {\textbf
R\textsuperscript{T}}, i.e. the inverse is equal to the transpose) with
all the entries of the matrix equal to zero except for the diagonal and
two off-diagonal elements in the positions $kl$ and $lk$ of the matrix,
as shown below.

\[
 \mathbf{R} =
\begin{pmatrix}
1 & 0 & 0 & \hdots & 0 & 0 \\
0 & 1 & 0 & \hdots & 0 & 0 \\
0 & \hdots & R_{k,k} & \hdots & R_{k,l} &  0 \\ 
\vdots & \vdots & \vdots & \ddots & \vdots & \vdots \\
0 &  \hdots & R_{l,k} & \hdots & R_{l,l} & 0 \\
0 & 0 & \hdots & 0 & 0 & 1 \\ 
\end{pmatrix}
\]

When a similar transformation is applied over the matrix
{\textbf A}, the off-diagonal elements of the new matrix {\textbf A\textsuperscript{*}} are equal
to zero, meaning that 
{A\textsuperscript{*}\textsubscript{kl}} = {A\textsuperscript{*}\textsubscript{lk}} = 0.

The idea of the algorithm is to find the largest off-diagonal element of
the matrix \textbf{A}, apply a rotation involving the row and column of
the largest element and save the rotation matrix \textbf{R}.
%
The rotations are applied until all the off-diagonal elements are lower
than a delta.
%
The application of the rotation matrix \textbf{R} over the matrix
\textbf{A} produces the new matrix {\textbf A\textsuperscript{*}}, whose
elements are given by

\begin{equation}\label{1} A^{*}_{kk} = A_{kk}  -  t A_{kl} \end{equation}
\begin{equation}\label{2} A^{*}_{ll} = A_{ll} +   t A_{kl} \end{equation}
\begin{equation}\label{3} A^{*}_{kl} =A^{*}_{lk} = 0       \end{equation}
\begin{equation}\label{4}
A^{*}_{kj} =A^{*}_{jk} = A^{*}_{kj} - s (A_{lj} + \tau A_{kj}), j \not = k \wedge j \not = l 
\end{equation}
\begin{equation}\label{5}
A^{*}_{lj} =A^{*}_{jl} = A^{*}_{lj} + s (A_{kj} - \tau A_{lj}), j \not = k \wedge j \not = l 
\end{equation}

where $s$, $t$ and $\tau$ are functions of A\textsubscript{kl}.

Once all the rotations are applied, the eigenvalues are the diagonal
elements of the final \textbf{A\textsuperscript{*}} and the eigenvectors
{\bf EV} are columns of the matrix product over all the Jacobi rotation
matrices.
%
\[ \mathbf{EV} = \prod_{i=1} \mathbf{R}_i \]
%
Because the rotation matrices are sparse, a partial product can be
calculated in each rotation step through the following transformation,
%
\begin{equation}\label{6} R^{*}_{jk} = R_{ik}  -  s (R_{jl} + \tau R_{jk}) \end{equation}
\begin{equation}\label{7} R^{*}_{jl} = R_{il}  +  s (R_{jk} - \tau R_{jl}) \end{equation}
%
where {\textbf R\textsuperscript{*}} denotes the partial product matrix.

\section{Haskell Implementation}

The repa library offers efficient operations over arrays; the data
structures and the functions of this library will be the basis for our
implementation.
%
You can find a repa tutorial here~\cite{repa}.

\begin{lstlisting}[float,captionpos=b,belowcaptionskip=4pt, caption= Jacobi Method]
import Data.Array.Repa  as R

type EigenValues = VU.Vector Double
type EigenVectors = Array U DIM2 Double

data EigenData = EigenData {
             eigenvals :: !EigenValues
           , eigenvec  :: !EigenVectors } deriving (Show)

jacobiP ::  (Monad m,VU.Unbox Double) =>
            Array U DIM2 Double  ->
            m LA.EigenData
jacobiP !arr = let (Z:. dim :. _dim) = extent arr
                   tolerance = 1.0e-9
               in  jacobi arr (LA.identity dim) 0 tolerance

jacobi :: (Monad m, VU.Unbox Double)
          => Array U DIM2 Double
          -> Array U DIM2 Double
          -> Step
          -> Tolerance
          -> m EigenData
jacobi !arrA !arrP step tol

  | step > 5*dim*dim = error "Jacobi method did not converge "

  | otherwise = case abs maxElem > tol of
       True -> do
               arr1 <- rotateA arrA (matrixA arrA args)
               arr2 <- rotateR arrP (matrixR arrP args)
               jacobi arr1 arr2 (step+1) tol

       False -> return $
                EigenData (diagonalElems arrA) arrP

  where (Z:. dim :. _dim) = extent arrA
        sh@(Z:. k :. l) = maxElemIndex arrA
        maxElem = arrA ! sh
        args = parameters maxElem aDiff k l
        aDiff = toval (l,l) - toval (k,k)
        toval (i,j) = arrA ! (Z:. i :. j)
\end{lstlisting}

Since the matrix is symmetric, we can work with either the upper or
lower triangular matrix.
%
Then both repa unidimensional unboxed arrays and bidimensional arrays
duplicating the data are suitable choices to represent our matrix.
%
We have chosen the bidimensional representation.

The main function has the signature depicted in Listing~1,
where the Jacobi function takes as input a bidimensional array
representing the symmetric matrix {\textbf A}, a bidimensional array for
the rotational matrix {\textbf R}, the current iteration (an integer) and
the numerical tolerance (which is just a synonym for a double).
%
The function returns an algebraic data type containing the eigenvalues
and eigenvectors, represented as a unboxed vector and a repa
bidimensional matrix, respectively.
%
The \textit{jacobiP} function is the driver to initialize the rotation
procedure, using the identity matrix as the initial value of the matrix
\textbf{R}.

The first guard in the Jacobi function takes care of the maximum number
of rotations allowed, where \textit{dim} is the number of rows (or
columns) of the symmetric matrix.
%
The second guard checks that the greatest off-diagonal element of the
symmetric matrix is larger than the tolerance.
%
If it is not, then the matrix is considered diagonalized and we return
an \textit{EigenData} value containing the eigenvalues in the diagonal of
the symmetric matrix \textit{arrA} and the final rotation matrix
contained in \textit{arrP}.

Parallel computation on arrays in repa is abstracted using a
generic monad \textit{m}, as stated in the signature of the Jacobi
function; therefore, \textit{rotateA} and \textit{rotateR} are monadic
functions.
%
Taking advantage of syntactic sugar, we extract the two new rotated
matrices \textit{arr1} and \textit{arr2} and bind them to a new call of
the Jacobi function.
%
For calculating the \textit{k} and \textit{l} indexes, the
\textit{maxElemIndex} function finds the largest index of the
bidimensional array.
%
Finally, the \textit{parameters} functions compute an algebraic data
type containing the numerical parameters required for the rotation
functions.

\begin{lstlisting}[float,captionpos=b,belowcaptionskip=4pt, caption= rotateA function]
rotateA :: (Monad m ,VU.Unbox Double) =>
           Array U DIM2 Double ->
           (Int -> Int -> Double) ->
           m(Array U DIM2 Double)           
rotateA !arr !fun =
  computeUnboxedP $ fromFunction (extent arr)
                  $ ( \sh@(Z:. n:. m) ->
                    case n <= m of
                         True -> fun n m
                         False -> arr ! sh)

matrixA :: VU.Unbox Double =>
           Array U DIM2 Double ->
           Parameters ->           
           Int -> Int -> Double
           
matrixA !arr (Parameters !maxElem !t !s !tau !k !l) n m
  | (n,m) == (k,l) = 0.0
  | (n,m) == (k,k) = val - t*maxElem
  | (n,m) == (l,l) = val + t*maxElem
  | n < k && m == k = val - s*(toval (n,l) + tau*val)
  | n < k && m == l = val + s*(toval (n,k) - tau*val)
  | k < m && m < l && n == k = val - s*(toval (m,l) + tau*val)
  | k < n && n < l && m == l = val + s*(toval (k,n) - tau*val)
  | m > l && n == k = val - s*(toval (l,m) + tau*val)
  | m > l && n == l = val + s*(toval (k,m) - tau*val)
  | otherwise = val

  where val = toval (n,m)
        toval (i,j) = arr ! (Z :. i:. j)
                   
\end{lstlisting}

Listing~2 contains the implementation of \textit{rotateA}.
%
The key piece of the rotation implementation is the
\textit{fromFunction} function, which is included in the repa library
and has the following signature 
\textit{fromFunction :: sh -> (sh -> a) -> Array D sh a}.
%
This function creates an array of a given shape from a function that
takes as an argument an index of an entry in the new array, and
calculates the numerical value for that entry.
%
The result is a ``delayed'' array which can be evaluated in parallel
using the \textit{computeUnboxedP} function.
%
Taking advantage of the symmetric properties of the matrix, we can
rotate only the upper triangular matrix and leave the rest of the elements untouched.
%
Therefore, we pass to \textit{rotateA} a partially applied
\textit{matrixA}, which takes the indexes \textit{m} and \textit{n} for
an upper triangular matrix and generates the numerical values using
equations \eqref{1} to \eqref{5}, leaving the values below the diagonal
untouched.

The implementation of \textit{rotateR} only differs from the previous
one, in that equations \eqref{6} and \eqref{7} are used to calculate the
numerical values and that the whole matrix is rotated not only the
triangular part, as depicted in Listing~3.

\begin{lstlisting}[float,captionpos=b,belowcaptionskip=4pt, caption= rotateR function]
rotateR :: (Monad m ,VU.Unbox Double) =>
           Array U DIM2 Double ->
           (Int -> Int -> Double) ->
           m(Array U DIM2 Double)
rotateR !arr !fun =
  computeUnboxedP $ fromFunction (extent arr)
                  $ ( \sh@(Z:. n:. m) -> fun n m)
        
matrixR :: VU.Unbox Double =>
           Array U DIM2 Double ->
           Parameters ->
           Int -> Int -> Double
matrixR !arr (Parameters !maxElem !t !s !tau !k !l) n m
  | m == k = val - s*((toval (n,l)) + tau*val)
  | m == l = val + s*((toval (n,k)) - tau*val)
  | otherwise = val

  where val = toval (n,m)
        toval (x,y) = arr ! (Z:. x :. y)
\end{lstlisting}

\subsection{Performance: when to be lazy}

As we already know, Haskell is a non-strict language, where major implementations (for instead GHC)
use a strategy called call-by-need or laziness to evaluate the code~\cite{Laziness}. 

There is a slight difference between laziness and non-strictness. Non-strict
semantics refers to a given property of Haskell programs that you can rely on: 
nothing will be evaluated until it is needed. The way we apply this strategy to 
our code is by using a mechanism called lazy evaluation. Lazy evaluation is the 
mechanism used by Haskell to implement non-strictness, using a device called thunk. 

Laziness can be a useful tool for improving performance on large arrays as one would deploy
schemes that do not need to evaluate all array members to compute certain matrix operations,
but in the general case where most matrix values will be taken in account it will 
reduce performance by adding a constant overhead to everything that needs to be evaluated.

And because of laziness, normally the compiler will not evaluate function arguments and pass 
the values to the function as in most languages with call-by-value strict semantics, 
so it will record the expression in the heap in a suspension structure called thunk in case it is 
evaluated later by the function. 

Storing and then evaluating most thunks is costly, and unnecessary in this case, when we know 
most of the time the complete array of values needs to be fully evaluated. So we will change the
way some parts of the code behave forcing strictness where we know its better.
Optimising compilers like GHC yet try to reduce the cost of laziness using strictness analysis
 ~\cite{Strictness}, which attempts to determine if a function is strict in one or more of 
its arguments, (which function arguments are always needed to be evaluated before entering the function). 
Sometimes this leads to better performance where the compiler can hint itself on what is worth to evaluate
strictly and what is not, but sometimes the programmer knows better what its worth 
to evaluate beforehand.

With bang patterns, we can hint the compiler about strictness on any binding form, 
making the function strict in that variable. In the same way that explicit type annotations can 
guide type inference, bang patterns can help guide strictness inference. Bang patterns 
are a language extension, and are enabled with the Bang Patterns language pragma.

Haskell also has the ability to manage strictness issues using the type system. We took advantage of such 
capability and defined strictness for some of our data types, thus allowing to force evaluation whenever
 those types are used. You can see that we also used unboxed variants of some the vector library,
as those types are carefully coded to keep them unboxed most of the time. You can see some of our strict data
types in Listing 4, following the suggestion given by the repa authors~\cite{bang}.

\begin{lstlisting}[float,captionpos=b,belowcaptionskip=4pt, caption= Strict data types for eigenvalue operations]

type EigenValues = VU.Vector Double
type EigenVectors = Array U DIM2 Double
data EigenData = EigenData {
             eigenvals :: !EigenValues
           , eigenvec :: !EigenVectors } deriving (Show)

\end{lstlisting}

So usually (at least while humans still can surpass compiler built-in intelligence) we can decide among several 
ways to control strictness on certain functions or data structures. we have seen that Haskell has easy ways to do
 this allowing to the programmer to tell the compiler where to be eager 
about evaluating things. As we have seen before Jacobi's method its a recursive algorithm that
 attempts to converge values below a certain threshold 
in order to compute the desired {\textbf A\textsuperscript{*}} matrix.

As we are using recursion we keep passing arguments every iteration and we need to ensure those arguments will be evaluated just before 
we pass them, avoiding to carry thunks along the way. The GHC bang pattern extension allow us to force evaluation on those arguments, as shown
in Listing 5.

\begin{lstlisting}[float,captionpos=b,belowcaptionskip=4pt, caption= Jacobi strict argument passing]
jacobi !arrA !arrP step tol
...
...
               arr1 <- rotateA arrA (matrixA arrA args)
               arr2 <- rotateR arrP (matrixR arrP args)
               jacobi arr1 arr2 (step+1) tol
...
...
\end{lstlisting}

Generally, in the implementation of laziness the thunk is really just a pointer to a piece of (usually static) code, plus another pointer 
to the data the code should work on. If the entity computed by the thunk is larger than the pointer to the code and the associated 
data, then the thunk wins out in memory usage. But if the entity computed by the thunk is smaller, build the thunk ends up using 
more memory that trying to evaluate the value eagerly.

So we are going to force values to avoid those costly thunks while we manage to rotate our arrays, 
converting them to strict forms ensuring that those operations are carried out before we use the results,
and also in the spirit of the previously repa works we hope the conjunction of strict values 
and tight loops will guide the compiler on the way of generating unboxed values as much as it is desired.

\subsection{Performance: Exploiting muti-core}

Ab-initio electronic structure methods like Hartree-Fock, as we will see,
 have the advantage to give accurate solutions, if the approximations 
are sufficiently small in magnitude. The downside of ab-initio methods
 is their computational cost. They often take enormous amount of time, 
memory and disk for describing the electrons, as we will see in the next section. The HF method scales nominally 
as {\textbf N\textsuperscript{4}}, with N being the number of electrons in the system. 


However in practice it can scale closer to {\textbf N\textsuperscript{3}} using heuristics,
for instance some of the integrals that will be introduced in next sections are zero or extremely
small and they are identified and neglect before the evaluation. Therefore we can employ this and some other
schemes to alleviate costs through simplification of some multi-index integrals that reduce electronic densities. 
Some of this techniques means diverse trade-off, but can sharply reduce the effective molecular size
in term of cost, a major problem in the treatment of biologically-sized molecules.

Despite the many methods that one can depict to alleviate complexity,
 we must in first term try to squeeze the most computing power out of our machines
and that is why there are people attempting to spread the computations across several cores.
 Indeed we will not stop here, in the advent of the new Erlang-alike 
distribution capabilities that Haskell begins to enjoy, we want to promote this new style of
 coding as a better way of exploiting cloud computing among the 
Computational Chemistry community.

So we are concerned with the matrix manipulation routines in terms of multi-core computation,
and this is where the repa shines, combined with strategies to build the source data in parallel fashion. In this way,
we can enjoy the full spectrum of Haskell computational capabilities, using parallelism with strategies and
concurrency (threads) on the repa side.

All operating systems provide threads that work reasonably well, but they are far too expensive in terms 
of resource management. Typical operating systems struggle to handle thousands of threads, whereas current state of 
affairs requires to handle a large amount of them, well over hundred of thousands even millions of them, 
provided that you will probably find that multi-core machines started to plague all computing
niches even many core computing is commodity nowadays.

Moreover green threads, otherwise known as lightweight threads
or user-space threads, are a well-known technique for avoiding the overhead 
of operating system threads. The idea is that threads are managed by the 
program runtime itself, or a library (in the case of certain languages 
that do not posses a process/thread abstraction), rather than by the operating system itself. 
Managing threads in user space should be cheaper, because fewer traps into 
the operating system are required but the drawback is that most user space threads 
can not be scheduled into the cores, in most operating systems.

These challenges not only concern to Haskell, we can see similar approaches
in other languages, being Erlang one of the most prominent among the hall of 
fame of massive concurrency promises. GHC designers are not behind when it 
comes to solutions to the multi-core problems. These challenges have Brought their
own solutions tied to the particular nature of the Haskell language,
 using a intelligent mixture of Green threads and OS threads,
as explained ~\cite{Peyton}.

What that means when it comes to the repa library?, well Repa introduced
delayed arrays that allows us to fuse multiple array operations, and 
minimise the overhead of index-space transformions, that are one of the most
common operations. Delayed arrays are represented by a function from indices 
to array elements as we have seen before, in contrast with manifest arrays
which are represented as contiguous blocks of unboxed values that other 
languages like Fortran traverse all the time. 

Fusion of operations on delayed arrays amounts to function composition, and 
can dramatically save lot of computational cost while being easy to manage. 
Delayed arrays can conveniently forced into manifest ones within repa by means 
of a loop. In a Fortran or C program, the programmer writes explicit loops. 
In Repa, the programmer never writes loops; the loops are in library functions.

Moreover, Repa also introduced a more advanced array layout with delayed or manifest 
arrays becoming a compound of extents and regions with multiple index functions, 
allowing efficient stencil convolutions.

Finally, repa makes use of the Haskell threads capabilities, and allows to optimize and
distribute the calculation of the array elements in one or more tight loops running concurrently.


\subsection{Benchmark}

In order to establish a baseline to measure the performance of our code, 
we use the Python code because this allow us compare between ``near C speed code``
(Numpy) with the repa implementation. We developed a test framework consisting 
on up to twenty independent runs for every Jacobi implementation in Python and Haskell.

Every test run consists on an initial step that loads a text file containing
the matrix to be diagonalized (we use a 100x100 matrix), then some code parses
this file and transform the data into a suitable form, for the  function. 
For the Python Numpy implementation we have chosen the built in Numpy array type
as the data structure to manipulate the matrices. Finally the diagonalization is
carried out, forcing the evaluation in the Haskell code to mimic the strict 
semantics in the Python side. Our test bed was a Intel core i5 @2.5 GHz 
laptop with 8GB RAM and OpenSuSE 11.4 x64 in which our Python code 
ran on an average value of 60.2 seconds. 

as we have little expertise with Haskell semantics and 
how to fine tune Garbage collection so we did not tried at first to outperform the Python
code, despite that repa seems to provide C speed performance, (consult Listings 15 onwards, in the appendix).
Despite that, We get near Python times with very little effort, but we were concerned about garbage collection 
issues, as our code showed consistently several seconds of garbage collection activity. Provided that Haskel
delivers parallel garbage collection (from GHC 7.0 onwards) we tried to perform as much memory management in parallel
as mutation activity, in order to free mutator threads (running repa threads mainly) from garbage related work.

From GHC users manual ~\cite{ghc} we found some experimental switches to allow the RTS to perform such activities in parallel
with user code and also the possibility of performing parallel garbage collection only on younger generations.
We tried to see if this setup would allow repa threads to run concurrently with garbage collection without  
disrupting each other.

Being almost newcomers on this arena, we still are not certain about what is going on, but in the end we manage to low the 
running times (mostly by lowering garbage collection times) but this is a matter we will work in the future 
(we are afraid of that), as we are planning to build more routines in the future. Therefore we will provide
 criterion based benchmarking facilities in our cabal package to allow readers to test and validate our measurements.

\section{The Hartree-Fock Method}

Once we are familiarized with the repa set of utilities and having
a function for solving the eigenvalue problem, we are the position to talk about
Hartree-Fock. In the beginning of the previous century it was discovered
that the energy of physical systems is quantized (contradicting our intuition
that it must be a continuous), then all the scientific community had no other choice that accept
the mathematical beauty of the quantum theory, well supported by all the experiments~\cite{quantum}.
 Equipped with this theory we can study whatever molecular system,
if we are able to solve the Schr\"{o}dinger equation!~\cite{Schrodinger}.
Then it began the race to develop approximate methods for solving the Schr\"{o}dinger equation and
finally, in those mythological times, the Hartree-Fock method became
the basis methodology upon which more accurate methods were developed~\cite{hartree-fock}.
 
\par By the middle of the previous century, the first programs were written to solve the 
iterative equations that are the core of the Hartree-Fock method as we will see, using
the best language available at those times. The incredible thing is that the scientific community, even until
this day, neither wanted nor understood the possibilities of changing the algorithms to 
more powerful and expressive language, arguing the giant cost both in investment and programming time. Therefore 
there is still an irrational and cruel practice spread in many universities for punishing Ph.D. students
in physical and chemical departments, consisting in debugging thousand of lines of code in Fortran 77,
bad written and worse documented. Not because many people have spoken about how to implement
these things, nor because is written in many books or have been taught 
for many generations, it means that it is the only and best way of programming scientific code.

\par The idea of the method is to solve the time-independent Schr\"{o}dinger
 equation that can be formulated as
\[ \mathbf{H}\Psi = E\Psi \]

Where \textPsi\ is the famous wave function and \textbf{H} is the Hamiltonian. 
Let's picture the previous equation this way, assuming that the wave function 
can be described through an analytical equation (for real systems we do not know the analytical form),
like the exponential function of -\textit{\textalpha x},
then the Hamiltonian carry out operations over the function, such as derive it two times 
then add up the previous derivative to the function itself.
That chain of operations should result in a number representing the energy
multiplying the wave function itself.
\par In quantum mechanics, the wave function contains all the possible system information that 
we may need, and the operators represent properties that we can measure (called observables),
extracting the information associated with the operator from the wave function.
In the case of the Schr\"odinger equation the Hamiltonian operator extracts the energy. 
In quantum mechanics there is not the concept of a trajectory, therefore we can not depict an electron
as a tiny sphere travelling in a given direction with a particular velocity,
see the uncertainty principle~\cite{uncertainty}. The only thing
we can say is that there is a probability to find an electron
around a specific region, therefore the operators provide us with mean values of this properties which are
well correlated with experiments. 

\par The only problem with the Schr\"odinger equation as stated previously
is that we do not know how to solve it 
(actually, it is know but for the most trivial case),
then some approximations are introduced for bringing the equation into 
a formulation that it is solvable. As we are only concerned
in molecular system, the only relevant interactions are all the possible ones
between nuclei and electrons, then we
introduce the Born-Oppenheimer approximation which states that as the mass of the nuclei is several 
orders of magnitude larger than the electron mass, it is possible to consider that electrons move in the 
field of fixed nuclei. In other words, the nuclei are so heavy with respect to the electrons that the last ones move 
at much shorter time-scale than the nuclei. So from now on we will only be interested in solving 
the part of the system than involves only the electrons. Do not run away, we are almost ready for having fun.

\par To approximate the electronic wave function indicated as \textPhi\textsubscript{elec}
we will use a productory of monoelectronic functions, each monoelectronic
function (actually, the square of it) gives us an idea of the probability to find an electron around 
a nucleus. Each of these functions depends on the coordinates of the electron which represents and depends also 
parametrically on the coordinates of the nucleus around which is most probable to 
find the electron. This electronic wave function is expanded as follows, 
\begin{equation}\label{8}
 \Phi_{elec}(\mathbf{r_{1}},\mathbf{r_{2}},...,\mathbf{r_{n}}) =
 \chi_{1}(\mathbf{r_{1}})\chi_{2}(\mathbf{r_{2}})...\chi_{n}(\mathbf{r_{n}})
 \end{equation}                                                                                                                
where r\textsubscript{i} is the coordinate of the nth electron.

And then we can redefine the electronic Schr\"odinger equation as a set of n-coupled equations of 
the form 
\begin{equation}\label{9}
 f_{i}\chi_{i}(\mathbf{r_{i}}) = \epsilon_{i}\chi_{i}(\mathbf{r_{i}})
\end{equation}
where \textit{f\textsubscript{i}} is the so called Fock operator which is made up
of three operators, 
\begin{equation}\label{10}
\hat f_{i} = \hat T_{i}  + \hat V_{i} + \hat V^{HF}_{i} 
\end{equation}

The first term in the Fock operator takes into account the kinetic energy, the second
the electronic interactions between nuclei and the \textit{ith} electron and the last term
is the expression for the interaction between the \textit{ith} electron 
and all other electrons.

\subsection{The Basis Set}
For representing the monoelectronic functions of equation \eqref{8} a set of Gaussian 
functions is usually used. We will refer to the list of Gaussian functions for 
expanding the monoelectronic functions as the basis set. The Gaussian
functions have the form,
\begin{equation}\label{11}
\phi(\mathbf{R},\alpha,l,m,n) = x^{l}y^{m}z^{n} e^{-\alpha\mathbf{R^{2}}}
\end{equation}
Every basis set depends on of the nuclear coordinates around
which the expansion is made denoted by \textbf{R}. Each monoelectronic 
function is expressed as linear combination of M Gaussian 
functions each of them multiplied by a coefficient,
\begin{equation}\label{12}
\chi_{i} = \sum_{\mu = 1}^{M} C_{\mu i} \phi_{\mu}
\end{equation}
This expansion should contain infinite terms, in order to 
fully describe the function expanded. But if we
want to compute something at all, we should choose a finite basis.

\subsection{The Roothaan-Hall Equations}
In this section we will ask you, dear Haskeller, an act of faith. Please believe us,
that once you have introduced a basis set you can transform the set of equations into
\eqref{9} a matricial problem of the form 
\begin{equation}\label{13}
\mathbf{FC} = \mathbf{SC\epsilon}
\end{equation}
where now, the Fock \textbf{F} operator has a matricial representation, 
\textbf{C} is a matrix containing the coefficient of \eqref{12}, \textbf{\textepsilon}
is a diagonal matrix containing the energies for every equation like \eqref{9} and finally
the \textbf{S} matrix called the overlap matrix, whose meaning will be discussed below.

\par Matrices representing operators are Hermitian matrices, which are the generalization
of symmetric matrix with complex entries~\cite{hermitian}. But do not worry about it, our representation
contained only real entries and therefore are symmetric matrices. The Fock matrix entries 
are given by the following set of integrals,
\[ \int{\phi^{*}_{\mu}(\mathbf{r_{i}}) \mathbf{\hat F} \phi_{\nu}(\mathbf{r_{i}}) \mathbf{dr_{i}}} =
 \mathbf{F}_{\mu \nu} \]
Paul Dirac introduced a shorter and elegant notation for 
this kind of integrals (see~\cite{bra-ket}). Using the Dirac notation, these integrals are rewritten as
\begin{equation}\label{14}
 \braket{\phi_{\mu} \mid \mathbf{\hat O} \mid  \phi_{\nu}}  = \mathbf{O}_{\mu \nu} 
\end{equation}
These notation will be adopted in the next section.

The overlap  matrix results as collateral effect of expanding the monoelectronic 
functions using a basis which functions are not orthogonal to each other. The nature 
of the overlap matrix can be visualized if you think about a 2-dimensional vector:
you can write any real 2-dimensional vector using a linear combination of the two
vectors (1,0) and (0,1), they are orthogonal to each other. But in the case 
of using a basis that is not orthogonal, non-linear terms will appear and 
it is not possible to represent the vector as a linear combination, 
but if you manage to normalize the basis in some way, a linear expansion can
be used with the new normalized basis. In the same fashion, if you make a linear expansion 
of a function in some basis, the functions of the basis must be orthogonal
to each other. Each element of the overlap matrix has the form show below.
\begin{equation}\label{15}
S_{ij} = \inftyint {dz \inftyint {dy \inftyint {\phi^{*}_{i} \phi_{j}dx}}} 
\end{equation}

A orthogonalization procedure makes one the elements for which \textit{i=j} in \eqref{14}, and
the rest of elements become zero. Now we will put all the pieces together in the implementation, 
even though some details will be introduced as needed.


\subsection{A First Guess: The Core Hamiltonian}
\begin{lstlisting}[float,captionpos=b,belowcaptionskip=4pt, caption= Operators definition]
type NucCoord = [Double]

data Operator =  T | V NucCoord
                 deriving Show

(<<|) :: (NucCoord,Basis) -> Operator ->  ((NucCoord,Basis),Operator)
b1 <<| op  = (b1,op)

(|>>) ::  ((NucCoord,Basis),Operator) -> (NucCoord,Basis) -> Double
(b1,op) |>> b2 = case op of
                T -> tijTotal b1 b2
                V rc -> vijTotal b1 rc b2

kinetic_12   = (r1,b1) <<| T |>> (r2,b2)
potential_12 = (r1,b1) <<| V r3 |>> (r2,b2)
\end{lstlisting}

In Listing 6, it is shown the infix notation that represents the Dirac notation:
every monoelectronic function over which the operator is applied,
is represented by a tuple containing the basis in which the function is expanded 
and the nuclear coordinates. Then, an algebraic data type is used for representing the
operators that make up the Fock operator. Using the two infix operators of Listing 6
we can squeeze the operators of \eqref{10} in the middle of the
two monoelectronic functions and we have a representation in Dirac notation, as exemplified
but the kinetic and potential expressions in Listing 6. We use the Dirac notation as
synonym for other function behind the scenes, helping with the readability of the code.


The matricial representation of the kinetic and electron-nucleus operators has 
an analytical solution, but for the interaction among the electrons we do not have 
an analytical representation for more than 3 electrons interacting among themselves, this is the so called
many-body problem~\cite{many-body}. Then for our first guess,
we applied a very human principle: if you do not know how to solve some problem, ignore it!
Hence, ignoring interactions among electrons we have our first representation for 
the Fock Matrix, as depicted in Listing 7.

\begin{lstlisting}[float,captionpos=b,belowcaptionskip=4pt, caption= Core Hamiltonian]

hcore :: [NucCoord] -> [Basis] -> [ZNumber] -> Nelec -> Array U DIM1 Double
hcore coords basis atomicZ nelec = 
 LA.list2ArrDIM1 dim (cartProd `using` parList rdeepseq)

 where dim = (nelec^2 + nelec) `div`2
       list = zip coords basis       
       cartProd = do
          (i,atomi) <- zip [1..] list
          (j,atomj) <- zip [1..] list
          guard (i<=j)
          let sumVij = foldl1' (+) . getZipList $
                       (\z rc -> ((-z) * atomi <<|Vij rc|>> atomj))
                         <$> ZipList atomicZ <*> ZipList coords
          return $ (atomi <<|Tij|>> atomj) + sumVij

\end{lstlisting}
 
\par Before going into details about this matrix which is called the
core Hamiltonian, let's take a look at its form. Below it is shown the equation
describing the entries of the core Hamiltonian:
\begin{equation}\label{16}
HCore_{ij} =\braket{\chi_{i} \mid \mathbf{\hat T} \mid \chi_{j}} + 
\sum_{k=1}^{N} \braket{\chi_{i} \mid \mathbf{\frac{1}{R_{k}}} \mid \chi_{i}} 
\end{equation}

Each element of the core Hamiltonian matrix is the sum of integrals represented through
the Dirac notation of \eqref{14}. This equation tells us that each element is composed
of the kinetic energy plus the summatory of interactions between one electron and all the N
nuclei that make up the molecule.

In agreement with the Dirac notation of Listing 6, in our implementation we represent the
monoelectronic function \textchi \textsubscript{i} with a tuple 
(\textbf{r\textsubscript{i}},basis), containing the nuclear coordinates 
and the basis for doing the expansion of \eqref{12}.

In Listing 7, it is shown the Haskell implementation of our first representation of 
the Core Hamiltonian. Since the matrix is a symmetric one we have decided to implemented it
as a unidimensional array containing the upper triangular matrix.
The function for calculating the matrix, requires the nuclear coordinates of all atoms, the basis used
for expanding the monoelectronic functions, the charge of each atom (the \textit{Znumber}),
this last number being necessary to calculate the electronic attraction 
between nuclei and electrons, and finally the number of electrons. Firstly, we
calculate the entries of the matrix as a parallel list exploding the parallel strategies,
triggering a spark for each element of the list representing the monoelectronic functions, 
(see more about strategies here :~\cite{strategies}). For taking maximal advantage
of the sparks system, a right level of granularity must be chosen, this means that 
each monoelectronic function which depends on a number of Gaussian functions, should contain
a minimal set in order to balance the charge in the processors if we want to be able to
speed up the calculation. This is a good thing, because in real calculations we
have very large basis sets.

After we have evaluated the list, using the auxiliary functions \textit{list2ArrDIM1}
and the dimension of the array, the list is transformed into an unboxed unidimensional repa array.
The building of the list containing the entries of the core Hamiltonian called \textit{cartProd},
takes advantage of the list monad. We first form a list of tuples representing the monoelectronic
functions zipping all the coordinates with their respective basis and then we generate the
indexes \textit{i},\textit{j} and the associated monoelectronic functions for those 
indexes in the core Hamiltonian matrix. Using a guard we ensure that only the indexes of
upper triangular matrix are taken into account. Then according to \eqref{17} we return
the result of applying the kinetic operator to two monoelectronic functions plus a summatory
which use the applicative style and the alternative applicative functor instance of the list functor, the
\textit{ZipList} instance. There is a lambda function that accepts two parameters, the atomic number Z and
the nuclear coordinates and return the desired interaction.  We first lift the lambda function applying it
to every element of the \textit{ZipList} which contains all the atomic numbers and then we apply the 
functor ZipList of partial applied functions to the functor \textit{ZipList} 
containing all the coordinates. Finally we fold over the final list after extracting
the result with \textit{getZipList}.

\subsection{The overlap matrix and the Jacobi Method}

In the previous section we have learnt how to build a first guess to the Fock matrix,
but for solving our target equation \eqref{13} we need to get rid of the overlap matrix.
A transformation for the overlap matrix is required in such a way that the overlap matrix
is reduced to the identity matrix as follows,

\begin{equation}\label{17}
 \mathbf{X^{\dagger}SX} = \mathbf{1}
\end{equation}
Where \textbf{1} is the identity matrix. 

\par The famous physicist Per-Olov L\"owdin, proposed the following transformation,
\begin{equation}\label{18}
\mathbf{X} = \mathbf{S^{-\frac{1}{2}}} 
\end{equation}

Because \textbf{S} is an Hermitian matrix  \textbf{S\textsuperscript{-1/2}} is Hermitian too,
which means that the transpose of the matrix, is the matrix itself (it the matrix is
symmetric the transpose matrix is equal to the matrix itself)
\[ \mathbf{{S^{-\frac{1}{2}}}^{\dagger}} = \mathbf{S^{-\frac{1}{2}}} \]
then
\[ \mathbf{S^{-\frac{1}{2}}\ S\ S^{-\frac{1}{2}}} =  \mathbf{S^{-\frac{1}{2}} S^{\frac{1}{2}}}
= \mathbf{S^0} = \mathbf{1}
\]
When it is applied the transformation in \eqref{14},
we get a new set of equations of the form
\begin{equation}\label{19}
\mathbf{F'C'} = \mathbf{C'\epsilon} 
\end{equation}
where 
\begin{equation}\label{20}
\mathbf{F'} = \mathbf{X^{\dagger}FX} \text{ and }  \mathbf{C'} = \mathbf{X^{-1}C}
\end{equation}
Finally we arrive to a standard eigenvalue problem, but first we need to generate
the transformation of \eqref{17} that is called a symmetric orthogonalization. The
matrix \textbf{S\textsuperscript{-1/2}} can be visualized as the application of the 
function square root to the matrix \textbf{S}. For calculating a
function over a diagonal matrix we simply apply the function
over the diagonal elements. For non-diagonal matrices they should be first diagonalized, then
the function is applied over the diagonal elements. Therefore the \textbf{S\textsuperscript{-1/2}}
matrix is given by,
\begin{equation}\label{21}
\mathbf{S^{-\frac{1}{2}}} = \mathbf{U{s^{-\frac{1}{2}}}U^{\dagger}}
\end{equation}
where the lower case \textbf{s\textsuperscript{-1/2}} is a diagonal matrix. 

\par The Jacobi algorithm can be used for diagonalizing a matrix \textbf{M},
where the eigenvalues calculated are the entries of the diagonal matrix
and the eigenvectors make up the matrix that diagonalized \textbf{M},
which are denoted as \textbf{U} in \eqref{21}. 

\begin{lstlisting}[float,captionpos=b,belowcaptionskip=4pt, caption= Symmetric Orthogonalization]
import qualified LinearAlgebra as LA
import qualified Data.Vector.Unboxed as VU

symmOrtho :: (Monad m, VU.Unbox Double) 
             => Array U DIM2 Double 
             -> m (Array U DIM2 Double)
symmOrtho !arr = do
symmOrtho arr = do
  eigData <- jacobiP $ arr
  let eigVal = LA.eigenvals eigData
      eigVecs = LA.eigenvec  eigData
      invSqrt = VU.map (recip . sqrt) eigVal
      diag = LA.vec2Diagonal invSqrt
  eigVecTrans <- LA.transpose2P eigVecs
  mtx1 <- LA.mmultP eigVecs diag
  LA.mmultP mtx1 eigVecTrans
\end{lstlisting}

In Listing 8 is depicted the symmetric orthogonalization procedure to calculate
the \textbf{S\textsuperscript{-1/2}} matrix, the \textit{LinearAlgebra} module 
contains some subroutines tailored for making matricial algebra using repa.
 Some of the functions there are taken from the repa examples~\cite{repaExam},
the rest are based on the repa library functions. The \textit{symmOrtho} 
function only requires the overlap matrix, which is first diagonalized using 
the Jacobi algorithm, resulting in an algebraic data type containing the eigenvalues
 as a unboxed vector and the eigenvectors  as a bidimensional matrix. The \textit{eigenvals} 
and \textit{eigenvec} are accessor functions
for retrieving the eigenvalues and eigenvectors, respectively.
Then the inverse square root of the eigenvalues is taken and with the resulting vector
a new diagonal matrix is created using \textit{vec2Diagonal}. Finally, using
the repa-examples, functions \textit{transpose2P} and \textit{mmultP}, which are 
the transpose and the matrix multiplication functions, respectively. The diagonal
function is multiplied by the matrix containing the eigenvalues and by its transpose,
resulting in the desired \textbf{X} matrix of \eqref{18}.

Using the symmetric orthogonalization procedure and the Jacobi method, equations \eqref{19}
and \eqref{20} can be solved resulting in the energies for the first approximation.


\subsection{The Variational Method}

In the previous section we derived a first approximation for the time-independent 
Schr\"odinger equation, but unfortunately we can not ignore the interactions 
among electrons. It is not known an analytical formulation for the interaction
of many electrons (in general for many particles), therefore only 
interactions between pairs of electrons can be calculated and the
force acting over an electron is calculated approximately as
the average of the interacting pairs. The average is built using the coefficient
for expanding the monoelectronic functions of \eqref{12}. The average force
rises a fundamental question: how do we know that the chosen coefficients of \eqref{12}
are the best ones for approximating the interactions among the electrons? The variational
principle is the answer.

\begin{theorem}[Variational Principle]
Given a normalized function \textPhi, that vanish at infinity, then the expectation value 
of the Hamiltonian is an upper bound to the exact energy, meaning that
\[ \braket{\Phi \mid \mathbf{H} \mid \Phi} \geq \varepsilon_{0} \] 
\end{theorem}

This theorem states that if we have a function for representing \textPhi\textsubscript{elec},
the resulting energy after applying the Hamiltonian operator over the function is always greater
that the real energy. Because  \textPhi\textsubscript{elec} depends on the expansion coefficients of
 of \eqref{12}, if we vary those coefficients in a systematic way we can generate 
a better electronic wave function \textPhi\textsubscript{elec} and a more accurate value
for the energy.

In practice the variational theorem states that better \textPhi\textsubscript{elec} are
obtained varying recursively a trial function.

\subsection{The Contraction: Squeezing Dimensions}

the recursive procedure described previously required the inclusion of the operator
for describing the pair interactions between electrons. Then, the Fock Matrix can
be reformulated as,

\begin{equation}\label{22}
 \mathbf{F} = \mathbf{HCore} + \mathbf{G}
\end{equation}

\begin{lstlisting}[float,captionpos=b,belowcaptionskip=4pt, caption= Computation of the  G matrix]
import Data.Array.Repa  as R

calcGmatrix !density !integrals =
  computeUnboxedP $ fromFunction (Z:. dim)
                  (\(Z :. i ) ->  sumAllS $
                  fromFunction (Z :. nelec)
                    (\( Z:. l) ->
                    let vec1 = unsafeSlice density (getRow l)
                        vec2 = map2Array integrals sortKeys (i,l)
                              nelec
                    in sumAllS . R.zipWith (*) vec1 $ vec2 ))

  where getRow x = (Any :. (x :: Int) :. All)
        (Z:. nelec :. _) = extent density
        dim = (nelec^2 + nelec) `div` 2

\end{lstlisting}

Where the \textbf{G} term stands for the interactions between electrons.
This term depends on the coefficients matrix in \eqref{13},
and on two types of integrals associated with the interacting electrons 
(\textit{J} and \textit{K}, called the Coulomb and interchange integrals).
For giving an analytical expression to the previous term, let us define 
a matrix that is function of the coefficients used for expanding the
monoelectronic function, called the density matrix, whose elements
are given by
\begin{equation}\label{23}
 P_{\alpha \beta} = 2\sum_{i=1} C_{\alpha i} C_{\beta i}
\end{equation}
where the summatory is carried out over the number of electrons.

The elements of the  \textbf{G} matrix are given by,
\begin{equation}\label{24}
 G_{\alpha \beta} = \sum_{k=1}\sum_{l=1} P_{lk} * 
(\braket{\alpha\beta \mid kl} - \frac{1}{2} \braket{\alpha l \mid k \beta })
\end{equation}
where
\[J(\alpha,\beta,k,l) = \braket{\alpha\beta \mid kl} \]
\[K(\alpha,l,k\beta,) = \braket{\alpha l \mid k \beta}\]

In an imperative language the usual way for implementing the \textbf{G} matrix
is to nest four loops, using a four dimensional array for saving the \textit{J} and \textit{K}
integrals which depend on four indexes as shown in \eqref{23}, in such a way that manipulating 
the four indexes in the innermost function is a real nightmare. In Haskell, surely we can do
better using the abstraction power. 
\par Before deepen in this multidimensional sea, a  rearrangement of \eqref{24} can help us
to bring this equation to more familiar lands,

\begin{equation} \label{25}
G_{\alpha \beta} =
\sum_{l=1}
\begin{bmatrix}
 P_{l1} & P_{l2} & \hdots & P_{ln} \\
\end{bmatrix}
\bullet
\begin{bmatrix}
\braket{\alpha \beta \mid \mid 1 l} & \braket{\alpha \beta \mid \mid 2 l} & 
\hdots & \braket{\alpha \beta \mid \mid n l } \\
\end{bmatrix}
\end{equation}
where 
\begin{equation} \label{26}
\braket{\alpha \beta \mid \mid k l} = 
\braket{\alpha \beta \mid k l} - \frac{1}{2} \braket{\alpha l \mid k \beta} = 
J - K 
\end{equation}
Equations \eqref{25} and \eqref{26} tell us that an entry of the 
\textbf{G} matrix can be considered as a summatory over an array of 
dot products between vectors.

In Listing 9, it is shown the implementation for calculating the \textbf{G} matrix, which
fortunately is a symmetric matrix too.  We use the recommended strategy suggested 
for the repa authors, evaluating in parallel the whole array but using sequential
evaluation for the inner loops. With the previous notes in mind, we begin our journey from
the first \textit{fromFunction} which is in charge of building the whole array: we pass to
this function the dimension of the final array (which is an upper triangular matrix) and 
the function for building the elements. Notice that 
as the implementation is done using unidimensional arrays for representing triangular
matrices, the first index \textit{i} encodes the \textalpha\ and \textbeta\ indexes of
\eqref{25}, meaning that \textit{i} should be decoded as the index of a bidimensional 
array. According to equations \eqref{25} and \eqref{26} the first
\textit{sumAllS} function adds up all the dot products, the innermost \textit{sumAllS}
collects the elements of each dot product, while the repa \textit{zipWith} function carries out
the desire dot operation between the vectors. The first vector is simply a row of the
density matrix, the second vector deserves a detailed analysis. 

\begin{lstlisting}[float,captionpos=b,belowcaptionskip=4pt, caption= The Map to Array Function]
map2Array :: M.Map [Int] Double 
             -> ([Int] -> [Int])
             -> (Int,Int) 
             -> Nelec 
             -> Array D DIM1 Double
map2Array mapIntegrals sortKeys (i,l) nelec = 
  R.fromFunction (Z:.nelec)
     (\(Z:.indx) ->
       let coulomb  = LA.map2val mapIntegrals $ sortKeys [a,b,indx,l]
           exchange = LA.map2val mapIntegrals $ sortKeys [a,l,indx,b]
       in coulomb - 0.5* exchange)
                                                            
  where ne = nelec-1
        pairs = [(x,y) | x <- [0..ne], y <- [0..ne], x<=y ]
        (a,b) = pairs !! i
\end{lstlisting}
 
As stated previously, in the traditional implementation a four dimensional array
is chosen for storing the \textit{J} and \textit{K} integrals. But arrays are not
the best way for storing data which is query many times and not in order. A Map 
seems a more appropriated data structure for saving the integrals. Therefore 
we have chosen the \textit{Map} structure, using as key a list of the four
indexes for the integrals. Besides, the integrals have the following
symmetry, 

\begin{equation} \label{27}
\begin{split}
\braket{\alpha \beta \mid kl} = \braket{ \beta \alpha \mid kl} 
= \braket{\beta \alpha \mid lk} = \braket{\alpha \beta \mid lk} \\
= \braket{kl\mid \alpha \beta} = \braket{lk \mid \alpha \beta} 
=\braket{lk \mid \beta \alpha} = \braket{kl\mid \beta \alpha}
\end{split}
\end{equation}

Therefore we only need to calculate one of the eight integrals. But, a systematic
way should be selected for choosing the indexes of the integral to be
evaluated. The increasing order is a good election, then from the eight possible integrals
only the integral with the lowest indexes is calculated and stored in a map. 

In Listing 10 is depicted the implementation of the \textit{map2Array} function for
calculating the vector of integrals used in the computation of the \textbf{G} matrix. The
arguments of this functions are the map containing the integrals, a function 
for sorting the keys, two indexes provided for the \textit{calcGmatrix} function
and the total number of electrons. The two indexes are used for generating
the key of the desired integral. The first of these indexes encodes the \textalpha\ and 
\textbeta\ indexes of \eqref{24} and \eqref{25}, for decoding those indexes a list
of tuples representing the indexes of a bidimensional matrix is calculated,
then the \textit{ith} index of the unidimensional array corresponds to the indexes (\textalpha,\textbeta).
The second index corresponds to the row of the density matrix according to \eqref{25}.Finally
the \textit{map2val} function, which is a lookup function with some error reporting properties, 
retrieves the required key for the map of integrals and builds the numerical values of the vector.
Probably, some of you have been wondering why we have use a list of tuples for decoding 
the indexes instead of using the functions \textit{toIndex} and \textit{fromIndex}
provided by the class \textit{shape} of repa, The problem is that we are working with
unidimensional representation of diagonal matrices and we can not use this pair of functions.
If you are not sure of this, try to use the \textit{fromIndex} function in a flatten array representing
a diagonal matrix.

The \textit{map2Array} function returns a delay array for performance reasons: it is more efficient
to carry the indices of the elements, performance some operations with them and finally
evaluate the whole array than computing the array in each step~\cite{fusion}.

\subsection{The Self Consistent Field Procedure}

The variational method establishes the theoretical tool for computing the best
wave function recursively. We have discussed the pieces of the Hartree-Fock
method separately, now it is time to assemble the machinery. First a recipe of the 
whole algorithm should be formulated. The recipe states that the SCF procedure is 
as follows:
  
\begin{itemize}
\item Declare the nuclear coordinates, the basis set and the nuclear
charges of all atoms.
\item Calculate all the integrals.
\item Diagonalize the overlap matrix using equations \eqref{17} and \eqref{18}.
\item Compute a first guess for the density matrix.
\item Calculate the \textbf{G} matrix.
\item Form the Fock matrix adding the core Hamiltonian and the \textbf{G} matrix.
\item Compute the new Fock matrix \textbf{F'} using equation \eqref{20}.
\item Diagonalize \textbf{F'} obtaining \textbf{C'} and \textepsilon'.
\item Calculate the new matrix of coefficients \textbf{C} using \textbf{C} = \textbf{XC'}.  
\item Compute a new density matrix using the above \textbf{C} matrix and \eqref{23}.
\item Check if the new and old density matrix are the same within a tolerance.
\item Compute again the \textbf{G} matrix in the case of the falsehood of the previous item. 
\item Return the energies along with the Fock and the density matrices.
\end{itemize}

Now using the syntactic sugar of the monads, we can cook our Hartree-Fock cake. First 
a function can be set for collecting all the required data before forming the 
\textbf{G} matrix. In Listing 11 is the implementation of the \textit{scfHF} function acting
as collector of the required data and as interface with client codes asking
for Hartree-Fock calculations. the Algebraic data type containing the results 
is also shown.

\begin{lstlisting}[float,captionpos=b,belowcaptionskip=4pt, caption= The Interface function]
data HFData = HFData {
           getFock      :: !(Array U DIM1 Double)
         , getCoeff     :: !LA.EigenVectors 
         , getDensity   :: !(Array U DIM2 Double)
         , getOrbE      :: !LA.EigenValues
         , getEnergy    :: !Double} deriving (Show)

scfHF :: (Monad m, VU.Unbox Double)
         => [NucCoord] 
         -> [Basis]
         -> [ZNumber]
         -> Nelec
         -> m (HFData)
scfHF coords basis zlist nelec= do 
        let core = hcore coords basis zlist nelec
            density = LA.zero nelec
            integrals = calcIntegrals coords basis nelec           
        xmatrix <- symmOrtho <=< LA.triang2DIM2 $ mtxOverlap coords basis nelec
        scf core density integrals xmatrix 0 500

\end{lstlisting}

The strict algebraic data type \textit{HFData} stores the Fock matrix as a triangular matrix, 
the matrix of coefficients whose type \textit{EigenVectors} is a synonym for a bidimensional array,
 the density matrix. The eigenvalues of the equation \eqref{19}, which are called the orbital energies,
 are saved in data structure called 
\textit{EigenValues} which is a synonym for a unboxed vector and finally 
the total energy which is given by the following expression,
\begin{equation}\label{28}
E_{0} = \frac{1}{2} \sum_{i}\sum_{j} P_{ji}(HCore_{ij} + F_{ij})
\end{equation}
where \textbf{P} is the density matrix.

The scfHF is in charge of building the core Hamiltonian and calculate the map containing
the integrals for computing the \textbf{G} matrix (for the first guess of 
the density matrix the zero matrix is usually used). The evaluation of the integrals 
deserves its own discussion, we are not going to enter in any detail 
about the calculation of the integral but the curious reader is 
invited to take a look at the \textit{IntegralsEvaluation}
module in~\cite{AngelyFelipe}. Besides, this function calculates the 
\textbf{X} matrix using the overlap matrix according to equations \eqref{17}
and \eqref{18}, but previous to apply the symmetric orthogonalization the upper triangular matrix
should be reshape to a bidimensional symmetric matrix using the monadic function
called \textit{triang2DIM2}. Finally the function which carries out the recursive part of
the scf procedure is called. 

The \textit{scf} function is depicted in Listing 12: the function takes as arguments
the core Hamiltonian, the previous density matrix, the \textbf{X} matrix, the integer label
of the current step and the maximum number of allowed steps. In case of exceeding
the maximum number of steps, we want to finish immediately, therefore we leave the error
to blow up in our face. If the maximum number of steps is not exceeded, the Fock matrix
is calculated adding the core Hamiltonian and the \textbf{G} matrix together, this last matrix is
calculated using the old density and the map of integrals.

\begin{lstlisting}[float,captionpos=b,belowcaptionskip=4pt, caption= Self Consistent Field Function]   
scf :: (Monad m, VU.Unbox Double)
     => Array U DIM1 Double
     -> Array U DIM2 Double
     -> M.Map [Int] Double
     -> Array U DIM2 Double
     -> Step
     -> Int
     -> m(HFData)
scf !core !oldDensity !integrals !xmatrix step maxStep
                                                                   
   | step < maxStep = do
       fockDIM1 <- fock core oldDensity integrals
       hfData <- diagonalHF fockDIM1 xmatrix
       etotal <- variationalE core fockDIM1 oldDensity
       let newHFData = hfData {getFock=fockDIM1, getEnergy=etotal}
           bool =  converge oldDensity . getDensity $ newHFData
       case bool of
            True -> return newHFData
            False -> scf core (getDensity newHFData) integrals
                     xmatrix (step+1) maxStep

   | otherwise =  error "SCF maxium steps exceeded"

\end{lstlisting}

Now, according to the recipe, we need to generate a new matrix \textbf{F'} using
the \textbf{X} matrix and then resolve this standard eigenvalue problem obtaining
the energies as eigenvalues and a new matrix of coefficients as eigenvectors. In order to
do so, we have defined a \textit{diagonalHF} function defined in Listing 13. The \textit{newFock} 
term on this function simply chains together two monadic functions which first take 
the unidimensional Fock matrix and translate it to its bidimensional form, then apply
equation \eqref{20} to the Fock matrix generating a bidimensional Fock matrix call \textit{fDIM2}
which is diagonalized using the Jacobi method and then the new \textbf{F'} is reshape to
a unidimensional array to be stored in the record. For retrieving the eigenvalues and eigenvectors of
the resulting algebraic data type \textit{EigData}, we can use the arrow operator (\&\&\&)
 in conjunction with the two accessor functions. Finally we obtain the new matrix of
coefficients and the density. Because the total energy is not calculated in this point,
a zero is added to the value constructor. 

\begin{lstlisting}[float,captionpos=b,belowcaptionskip=4pt, caption= The DiagonalHF Function]   

diagonalHF :: (Monad m, VU.Unbox Double) 
           => Array U DIM1 Double
           -> Array U DIM2 Double 
           -> m(HFData)
diagonalHF fock1 xmatrix = do
   fDIM2 <-newFock
   f' <- LA.toTriang fDIM2
   eigData <- jacobiP fDIM2
   let (coeff,orbEs) = LA.eigenvec &&& LA.eigenvals $ eigData
   newCoeff <- LA.mmultP xmatrix coeff
   newDensity <- LA.calcDensity newCoeff
   return $ HFData f' newCoeff newDensity orbEs 0.0

  where newFock = (LA.unitaryTransf xmatrix) <=< LA.triang2DIM2 $ fock1

\end{lstlisting}

Once the record containing the Hartree-Fock data has been calculated and coming back
to the \textit{SCF} function, we are in position to calculate the total energy using
\eqref{28} and its implementation called the \textit{variationalE} function, shown in Listing 14.

\begin{lstlisting}[float,captionpos=b,belowcaptionskip=4pt, caption= The DiagonalHF Function]   

variationalE ::(Monad m, VU.Unbox Double) => 
                  Array U DIM1 Double -> 
                  Array U DIM1 Double -> 
                  Array U DIM2 Double  ->
                  m Double
variationalE core fockMtx oldDensity = 
  (0.5*)`liftM` do
  sumHF <- (R.computeUnboxedP $ 
           R.zipWith (+) core fockMtx) >>= \arr ->
           LA.triang2DIM2 arr
  result <- LA.mmultP sumHF oldDensity
  LA.tr result

\end{lstlisting}

Finally using the record syntax we introduce the
total energy and the Fock matrix before the diagonalization procedure,
because it is the one useful for further calculations and finally we 
check for the convergence criteria. Based
on the boolean return by the convergence function, it is decided if more
variations of the coefficients are necessary of if we are done.

\subsection{Final Remarks}

We are by far not Haskell experts, only new kids in the school. Therefore all
you feedback is more than good, please let us know your opinion about
this project and we will try to answer your questions as best as we can.

The code began as a challenge and playground for developing a big project 
in Haskell. After some months and for our own astonishment, we found that
apart from the tuning procedure for speeding up the calculation, we can
easily design pretty complex structure with little effort. Many lesson 
are still to be learnt, but Haskell powerful type system, besides the 
community support with hundreds of libraries are from our point of view what will make 
scientific software written in Haskell outstanding.

The SCF procedure described in this article it is not the most use in
the quantum chemistry packages due to convergence problems, instead a
 method called direct inversion in the iterative subspace or direct
inversion of the iterative subspace (DIIS) is used, this method is 
based in SCF described above, we are working on its implementation.

The set of modules making up the Hartree-Fock method, which will 
become a package in a near future, are not a real competition
for the electronic structure package found either in the market
or in the academic community like these ones: (most of them
written in Fortran)~\cite{software},
but as far as we know it is the first one implemented in a 
functional language. But unlike, the politics of one of the 
most famous software in computational quantum chemistry, we
will not demand you or ban you to use our code if you
compare the performance or the results of our code with some
other ~\cite{banned}. 

Only remains to thank you dear Haskeller, for following us to these 
lands, full of opportunities for applying the high abstraction level
of Haskell to the challenge of simulating the natural phenomena. And
Remember \textit{Just Fun ...or Nothing}. 

\[\mathbf{BEWARE\;FortranIANS!!!} \]


\[\mathbf{\hat{H}askell\: \Psi >>= \setminus E\: -> \Psi\: E }\]

\section{Appendix}

The following Listings depict several tests, performed to make comparison on diverese setting of the Haskell Run Time System 
against the base line Python variant using Numpy.

\begin{lstlisting}[float,captionpos=b,belowcaptionskip=4pt, caption=Haskell test run 1]
/> jacobi-bench +RTS -N4 -s -RTS

69,566,094,376 bytes allocated in the heap
  66,061,024 bytes copied during GC
    606,264 bytes maximum residency (3333 sample(s))
    101,984 bytes maximum slop
       5 MB total memory in use (1 MB lost due to fragmentation)

                    Tot time (elapsed) Avg pause Max pause
Gen 0   45088 colls,45087 par 26.25s 6.60s 0.0001s 0.0234s
Gen 1   3333 colls, 3333 par  4.31s  1.04s 0.0003s 0.0205s

Parallel GC work balance: 1.60 (7756716 / 4860339, ideal 4)

           MUT time (Elapsed) GC time (elapsed)
Task 0 (worker) : 16.00s (44.96s) 4.32s (6.09s)
Task 1 (worker) :  0.00s (51.05s) 0.00s (0.00s)
Task 2 (bound) :  19.18s (46.97s) 3.66s (4.08s)
Task 3 (worker) : 15.16s (40.21s) 8.39s (10.84s)
Task 4 (worker) : 17.02s (39.63s) 9.91s (11.42s)
Task 5 (worker) : 16.17s (40.24s) 9.15s (10.81s)

SPARKS:0 (0 converted,0 overflowed,0 dud,0 GC'd,0 fizzled)

INIT  time   0.00s (0.00s  elapsed)
MUT   time  88.41s (43.41s elapsed)
GC    time  30.57s (7.64s  elapsed)
EXIT  time   0.00s (0.00s  elapsed)
Total time 118.98s (51.05s elapsed)

Alloc rate  786,848,499 bytes per MUT second

Productivity 74.3% of total user, 173.2% of total elapsed

\end{lstlisting}

\begin{lstlisting}[float,captionpos=b,belowcaptionskip=4pt, caption=Haskell test run 2]
/> jacobi-bench +RTS -N2 -qg1 -k512K -H1M -s -RTS

63,002,489,096 bytes allocated in the heap
    53,857,224 bytes copied during GC
     2,368,704 bytes maximum residency (1431 sample(s))
       109,600 bytes maximum slop
             7 MB total memory in use (1 MB lost due to fragmentation)

                        Tot time (elapsed) Avg pause Max pause
Gen  0 77545 colls,   0 par 2.60s 2.58s 0.0000s 0.0006s
Gen  1 1431 colls, 1430 par 1.57s 0.77s 0.0005s 0.0021s
Parallel GC work balance: 1.20 (1257580 / 1051806, ideal 2)

                MUT time (elapsed) GC time(elapsed)
Task  0 (worker) :25.29s (53.63s) 1.06s (1.07s)
Task  1 (worker) :26.72s (54.09s) 0.61s (0.62s)
Task  2 (bound)  :15.72s (52.27s) 2.43s (2.44s)
Task  3 (worker) : 0.00s (54.71s) 0.00s (0.00s)
SPARKS: 0 (0 converted,0 overflowed,0 dud,0 GC'd,0 fizzled)

INIT    time    0.00s  (  0.00s elapsed)
MUT     time   67.65s  ( 51.36s elapsed)
GC      time    4.18s  (  3.35s elapsed)
EXIT    time    0.00s  (  0.00s elapsed)
Total   time   71.83s  ( 54.71s elapsed)
Alloc rate    931,191,462 bytes per MUT second
Productivity  94.2% of total user, 123.7% of total elapsed

\end{lstlisting}


\begin{lstlisting}[float,captionpos=b,belowcaptionskip=4pt, caption=Haskell test run 3]
/> bin/jacobi-bench +RTS -N2 -qg1 -k512K -H1M -s -RTS

63,002,282,072 bytes allocated in the heap
  53,544,912 bytes copied during GC
   2,368,704 bytes maximum residency (1432 sample(s))
    109,600 bytes maximum slop
        7 MB total memory in use (1 MB lost due to fragmentation)

                 Tot time (elapsed) Avg pause Max pause
Gen 0   76984 colls,   0 par  2.52s  2.49s   0.0000s  0.0005s
Gen 1   1432 colls, 1431 par  1.52s  0.74s   0.0005s  0.0013s

Parallel GC work balance: 1.20 (1260845 / 1050453, ideal 2)

           MUT time (elapsed)    GC time (elapsed)
Task 0 (worker) : 24.48s  (51.90s) 1.01s ( 1.01s)
Task 1 (worker) : 25.95s  (52.32s) 0.60s ( 0.60s)
Task 2 (bound)  : 15.46s  (50.56s) 2.35s ( 2.35s)
Task 3 (worker) :  0.00s  (52.92s) 0.00s ( 0.00s)

SPARKS: 0 (0 converted, 0 overflowed, 0 dud, 0 GC'd, 0 fizzled)

INIT  time  0.00s ( 0.00s elapsed)
MUT   time  65.80s ( 49.69s elapsed)
GC   time  4.04s ( 3.23s elapsed)
EXIT  time  0.00s ( 0.00s elapsed)
Total  time  69.85s ( 52.92s elapsed)

Alloc rate  957,423,828 bytes per MUT second

Productivity 94.2% of total user, 124.3% of total elapsed

\end{lstlisting}

\begin{lstlisting}[float,captionpos=b,belowcaptionskip=4pt, caption=Haskell test run 4]
/> ./bin/jacobi-bench +RTS -N2 -qg1 -k512K -s -RTS
Loading Jacobi algorithm test data symm.txt file...
Parsing Jacobi data...
Normal benchmark:
63,002,262,568 bytes allocated in the heap
  52,620,728 bytes copied during GC
   2,368,728 bytes maximum residency (1431 sample(s))
    109,600 bytes maximum slop
       7 MB total memory in use (1 MB lost due to fragmentation)

                 Tot time (elapsed) Avg pause Max pause
Gen 0   74693 colls,   0 par  2.47s  2.45s   0.0000s  0.0005s
Gen 1   1431 colls, 1430 par  1.27s  0.60s   0.0004s  0.0017s

Parallel GC work balance: 1.22 (1276879 / 1050446, ideal 2)

           MUT time (elapsed)    GC time (elapsed)
Task 0 (worker) :  23.49s  ( 49.02s)    0.83s  ( 0.83s)
Task 1 (worker) :  24.95s  ( 49.20s)    0.64s  ( 0.65s)
Task 2 (bound) :  15.47s  ( 47.69s)    2.16s  ( 2.16s)
Task 3 (worker) :  0.00s  ( 49.85s)    0.00s  ( 0.00s)

SPARKS: 0 (0 converted, 0 overflowed, 0 dud, 0 GC'd, 0 fizzled)

INIT  time  0.00s ( 0.00s elapsed)
MUT   time  63.79s ( 46.81s elapsed)
GC   time  3.74s ( 3.04s elapsed)
EXIT  time  0.00s ( 0.00s elapsed)
Total  time  67.54s ( 49.85s elapsed)

Alloc rate  987,532,156 bytes per MUT second

Productivity 94.5% of total user, 128.0% of total elapsed

\end{lstlisting}

\begin{lstlisting}[float,captionpos=b,belowcaptionskip=4pt, caption=Haskell test run 5]
/> ./bin/jacobi-bench +RTS -N3 -qg1 -k512K -s -RTS

63,013,258,200 bytes allocated in the heap
  38,828,464 bytes copied during GC
   2,897,696 bytes maximum residency (1115 sample(s))
    117,792 bytes maximum slop
       9 MB total memory in use (1 MB lost due to fragmentation)

                 Tot time (elapsed) Avg pause Max pause
Gen 0   52789 colls,   0 par  2.35s  2.33s   0.0000s  0.0005s
Gen 1   1115 colls, 1114 par  1.18s  0.40s   0.0004s  0.0095s

Parallel GC work balance: 1.27 (1046069 / 822802, ideal 3)

           MUT time (elapsed)    GC time (elapsed)
Task 0 (worker):  21.66s  (43.29s) 0.51s ( 0.56s)
Task 1 (worker):  16.63s  (43.27s) 0.57s ( 0.58s)
Task 2 (bound) :  15.87s  (41.86s) 1.98s ( 1.99s)
Task 3 (worker):  20.32s  (43.40s) 0.40s ( 0.45s)
Task 4 (worker):   0.00s  (43.85s) 0.00s ( 0.00s)

SPARKS: 0 (0 converted, 0 overflowed, 0 dud, 0 GC'd, 0 fizzled)

INIT  time  0.00s  ( 0.00s elapsed)
MUT   time  74.42s (41.12s elapsed)
GC    time  3.53s  ( 2.73s elapsed)
EXIT  time  0.00s  ( 0.00s elapsed)
Total time  77.95s (43.85s elapsed)

Alloc rate  846,724,495 bytes per MUT second
Productivity 95.5% of total user, 169.7% of total elapsed
\end{lstlisting}

\section{Acknowledgement}

We want to thank Marco Marazzi for his help
in the redaction of the paper and our advisor
Professor Luis Manuel Frutos for his patient
and thoroughly support, without
him we would had been lynched by now!


\begin{thebibliography}{12}

\bibitem{Skibinski}
  http://www.haskell.org/haskellwiki/Numeric\_Quest
\bibitem{eigenvalue}
  http://en.wikipedia.org/wiki/Eigenvalues\_and\_eigenvectors
\bibitem{Jacobi}
 http://en.wikipedia.org/wiki/Jacobi\_rotation
\bibitem{repa}
 http://www.haskell.org/haskellwiki/Numeric\_Haskell:\_A\_Repa\_Tutorial
\bibitem{Laziness}
  https://en.wikibooks.org/wiki/Haskell/Laziness
\bibitem{Strictness}
  https://http://www.haskell.org/haskellwiki/Performance/Strictness
\bibitem{bang}
  http://hackage.haskell.org/packages/archive/repa/3.2.2.2/doc/html/Data-Array-Repa.html
\bibitem{Peyton}
  http://www.aosabook.org/en/ghc.html
\bibitem{ghc}
  http://www.haskell.org/ghc/docs/latest/html/users\_guide/
\bibitem{quantum}
 http://en.wikipedia.org/wiki/Quantum\_mechanics
\bibitem{Schrodinger}
  http://en.wikipedia.org/wiki/Schr\"{o}dinger\_equation
\bibitem{hartree-fock}
  http://en.wikipedia.org/wiki/Hartree-Fock\_method  
\bibitem{uncertainty}
  http://en.wikipedia.org/wiki/Uncertainty\_principle
\bibitem{hermitian}
  http://en.wikipedia.org/wiki/Hermitian\_matrix
\bibitem{bra-ket}
  http://en.wikipedia.org/wiki/Bra-ket\_notation
\bibitem{many-body}
  http://en.wikipedia.org/wiki/Many-body\_problem
\bibitem{strategies}
  http://hackage.haskell.org/packages/archive/parallel/2.2.0.1/doc/html/Control-Parallel-Strategies.html
\bibitem{repaExam}
  http://hackage.haskell.org/package/repa-examples
\bibitem{fusion}
  http://www.cse.unsw.edu.au/~benl/papers/guiding/guiding-Haskell2012-sub.pdf
\bibitem{software}
  http://en.wikipedia.org/wiki/List\_of\_quantum\_chemistry\_and\_solid-state\_physics\_software
\bibitem{banned}
  http://www.bannedbygaussian.org/
\bibitem{AngelyFelipe}





\end{thebibliography}

\end{document}







